\documentclass[10pt,a4paper]{article}
\usepackage[a4paper,margin=0.55in]{geometry}
\usepackage[utf8]{inputenc}
\usepackage[T1]{fontenc}
\usepackage{enumitem}

\newcommand{\headerrow}[2]
{\begin{tabular*}{\linewidth}{l@{\extracolsep{\fill}}r}
	#1 &
	#2 \\
\end{tabular*}}

\pagestyle{empty}
\setlength{\tabcolsep}{0em}

\usepackage{hyperref}
\hypersetup{
    %colorlinks=true,
    linkcolor=blue,
    filecolor=magenta,
    urlcolor=cyan,
		pdfauthor={Amedeo Chiefa},
    pdftitle={Amedeo Chiefa - Post graduate student - Curriculum Vitae},
		pdfsubject={Work History and Experience of Amedeo Chifa},
		pdfkeywords={CV, Résumé, Work History, Amedeo Chiefa},
    % pdfpagemode=FullScreen,
    }

\begin{document}

% Name
\begin{center}
    {\LARGE \textbf{Amedeo Chiefa}} \\[0.1cm]
    {\large Postgraduate Student} \\
    \vspace{0.2cm}
    \small
    The Higgs Centre for Theoretical Physics, University of Edinburgh \\[0.1cm]
    JCMB, KB, Mayfield Rd - Edinburgh EH9 3JZ, Scotland \\
    \vspace{0.2cm}
    \href{mailto:amedeo.chiefa@ed.ac.uk}{amedeo.chiefa@ed.ac.uk} |
    \href{https://www.linkedin.com/in/amedeo-chiefa-b6a6bb231/}{LinkedIn Profile}\ |
    \href{https://github.com/achiefa}{GitHub}\ |
    \normalsize
\end{center}

% ---------------- Summary ----------------
\hrule
\vspace{-0.4em}
\subsection*{Summary}
\begin{itemize}[leftmargin=1em]
  \item[] I am a Ph.D. candidate in Theoretical Particle Physics at the University of Edinburgh. 
          My research focuses on the phenomenological aspects of particle interactions, with a 
          particular interest in quantum chromodynamics. My work combines theoretical models of 
          fundamental interactions with high-energy experimental data to uncover the internal structure
          of the proton in terms of quarks and gluons. I am actively involved
          in the development of computational tools required tu pursue my research goals. These tools leverage 
          machine learning and artificial intelligence techniques, and I have a keen interest
          in the theoretical foundations of deep learning. Proficient in coding, I actively pursue personal
          coding projects to enhance my technical skills.
\end{itemize}

% ---------------- Education ----------------
\subsection*{Education}
\begin{itemize}[leftmargin=1em]
  \item[] 
  \headerrow
		{\textbf{Ph.D in Theoretical Physics}}
    {11/2023 – expected 10/2027}
    Specialisation field: Theoretical Particle Physics\\
    The University of Edinburgh (UK) — Supervisor: Prof. Luigi Del Debbio
  \item[]
  \headerrow
    {\textbf{M.Sc. in Theoretical Physics} (110/110 cum laude)}
    {24/10/2023}
    LM-17 — Classe delle lauree magistrali in Fisica, D.M. 270/04\\
    University of Turin — Supervisor: Prof. Emanuele R. Nocera\\
    Thesis title: Towards Polarised Parton Distribution Functions at next-to-next-to-leading order
  \item[]
  \headerrow
    {\textbf{B.Sc. in Physics} (109/110)}
    {20/07/2021}
    L-30 — Classe delle lauree in scienze e tecnologie fisiche, D.M. 270/04\\
    University of Turin — Supervisor: Prof. Marco Regis\\
    Thesis title: Self-interacting dark matter: cross-section calculation
\end{itemize}

% ---------------- Publications ----------------
\subsection*{Publications}
\begin{itemize}[leftmargin=1em]
    \item[] 
      \textbf{Parton distributions confront LHC Run II data: a quantitative appraisal} \\
      A. Chiefa, M. N. Costantini, J. Cruz-Martinez, E. R. Nocera, T. R. Rabemananjara, J. Rojo, T. Sharma, R. Stegeman, M. Ubiali \\
      Published in: JHEP 07 (2025) 067 | e-Print: \href{https://arxiv.org/abs/2501.10359}{arXiv:2501.10359}
    \item[] 
      \textbf{Status and Developments in Polarised Parton Distribution Functions} \\
      A. Chiefa\\
      \href{https://pos.sissa.it/469/200/pdf}{\textit{PoS} DIS2024 (2025) 200}
    \item[]
      \textbf{Helicity-dependent parton distribution functions at next-to-next-to-leading order accuracy from inclusive and semi-inclusive deep-inelastic scattering data} \\
      V. Bertone, A. Chiefa, E. R. Nocera \\
      Published in: {\it Phys.Lett.B} 865 (2025) 139497 | e-Print: \href{https://arxiv.org/pdf/2404.04712}{arXiv:2404.04712v1}
\end{itemize}

% ---------------- Conference and Workshop ----------------
\subsection*{Participation in Conferences and Workshops}
\begin{itemize}[leftmargin=1em]
    \item[]
    \headerrow
      {\textbf{31$^{\textrm{st}}$ International Workshop on Deep Inelastic Scattering (DIS2024)}}
      {08/04/2024 -- 12/04/2024}
      Talk title: \href{https://lpsc-indico.in2p3.fr/event/3268/contributions/7472/attachments/5307/7968/DIS_2024_MAP_4_3.pdf}{Towards helicity-dependent parton distribution functions at NNLO accuracy}
\end{itemize}

% ---------------- Research Experience ----------------
\subsection*{Funding and Programmes for the research activity}
\begin{itemize}[leftmargin=1em]
    \item[] 
      \headerrow
        {\textbf{Participation in Erasmus+ Traineeship CALL 2022}}
        {01/03/2023 -- 31/05/2023}
      CEA Paris-Saclay, IRFU (FR) -- Supervisor: Dott. Valerio Bertone \\[0.1em]
      I participated in the CALL 2022 of the Erasmus+ Traineeship funded by the European Commission. 
      The activity focused on the determination of the polarised parton distribution functions of the 
      proton at NNLO within the MAP framework.
\end{itemize}

% ---------------- Collaborations ----------------
\subsection*{Participation in International Collaborations}
\begin{itemize}[leftmargin=1em]
    \item[]
      \textbf{\href{https://nnpdf.mi.infn.it}{NNPDF Collaboration}} \\
      Junior member since November 2023. The NNPDF Collaboration
      performs phenomenological studies in the field of Quantum Chromodynamics, and determines
      the so-called parton distribution functions of the proton. I contribute actively to the
      maintenance and development of the NNPDF codebase. Currently, I am also involved in two
      side projects, one of which has resulted in a publication.
    \item[]
      \textbf{\href{https://github.com/MapCollaboration}{MAP Collaboration}} \\
      Member since March 2023. The MAP collaboration performs phenomenological
      analyses to extract the multidimensional distribution of partons inside hadrons. The main
      focus of the collaboration is the determination of Transverse Momentum Distributions, 
      Generalized Parton Distributions, and Fragmentation Functions. With other three members of the
      collaboration, I developed the first codebase to extend the analysis to polarised parton distribution 
      functions. I am currently working on the extension of the codebase to achieve higher order accuracy
      in the theoretical predictions.
\end{itemize}

% ---------------- Teaching activity ----------------
\subsection*{Teaching activity}
\begin{itemize}[leftmargin=1em]
    \item[]
    \textbf{Tutor} in Principles of Quantum Mechanics, 2024/2025 academic year, first semester \\
    3rd year undergraduate students, School of Physics and Astronomy, University of Edinburgh (UK)
    \item[]
    \textbf{Tutor} in Computer Simulation, 2023/2024 academic year, second semester \\
    2nd year undergraduate students, School of Physics and Astronomy, University of Edinburgh (UK)
\end{itemize}

% ---------------- Skills ----------------
\subsection*{Technical skills}
\begin{itemize}[leftmargin=1em]
  \item[] \textbf{Programming Languages:} Python, C++, bash, Fortran
  \item[] \textbf{Software:} Mathematica, MATLAB, \LaTeX, Git
\end{itemize}

% ---------------- Language Skills ----------------
\subsection*{Language skills}
\begin{itemize}[leftmargin=1em]
  \item[] Mother tongue: \textbf{Italian}
  \item[] Other languages: \textbf{English} (C1), \textbf{French} (Basic)
  \item[] Certifications:\\ 
  \textbf{English}: IELTS Academic (7.5/9.0)
\end{itemize}

\vfill
\hrule
\begin{description}
  \item
\headerrow
  {{\textsubscript{Written in {\LaTeX} and generated using GitHub actions.}}}
  {{\textsubscript{@@version@@}}}
\end{description}

\end{document}
